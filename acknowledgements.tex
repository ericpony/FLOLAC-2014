\documentclass[dissertation]{subfiles}

\subfileheadercorrection

\begin{document}

\newcommand{\zhname}[1]{\smash{\raisebox{-.99pt}{\includegraphics{pdfimages/#1.pdf}}}}

\hypertarget{acknowledgements}{}\bookmark[dest=acknowledgements]{Acknowledgements}

\chapter*{Acknowledgements}
\markboth{Acknowledgements}{Acknowledgements}

%It is immensely difficult for me to even start writing this part of my dissertation --- however I try, I still don't think I can give a proper account of how people have supported me and how I am grateful for their support during my three and a half years of DPhil study.
%However, not being able to do it well is probably not a good reason for not doing it, so here it goes.

I tend to link the appearance of my supervisor Jeremy Gibbons with that of Professor Smith in PhD Comics, but, very fortunately, Jeremy's personality is just the opposite of Smith's.
I can't hope for a more supportive supervisor than Jeremy: he is always eager to help when I encounter various problems, quickly responds to my email even when he's away or on holiday, and patiently listens to me as I struggle to find the right word to use during our meetings.
It is a great reassurance that he is there to offer advice --- usually I can quickly get unexpected yet perfectly sensible and clear suggestions from him, which are really the kind of suggestions one needs.

Besides Jeremy, I am very lucky to meet Richard Bird, Kwok-Ho Cheung, Daniel James, Geraint Jones, Tom Harper, Ralf Hinze, José Pedro Magalhães, Maciej Piróg, Meng Wang, and Nicolas Wu, who were present at the Algebra of Programming group meetings.
I probably won't fully realise how precious these meetings are until I leave --- it's inspiring to be in these meetings where people having similar interests so enthusiastically and freely share and discuss their findings.

Just a few days after my arrival in Oxford, at the orientation programme for international graduate students I met Frank Chen \zhname{FC}, who had given me a lot of support since then.
I still remember those morning walks around Christ Church Meadow.
Another person I met in my first year is Jackie Wang \zhname{JW}.
Especially memorable are the days when we worked on our respective dissertations together in the office, with lunch at the Bangkok House.
I also enjoyed many friendly chats with Lihao Liang and Di Chen.

During my second and third year, I lived in 45 Marlborough Road with Yen-Chen Pan \zhname{YCP} and Eric Liang \zhname{EL}.
(We were joined by Nien-Ti Tzou \zhname{NTT} and Lilya Lee for a year and then Giovanni Bassolino and Chih-Suei Shaw \zhname{CSS} for the other year.)
With their kind consent and tolerance, I had the luxury of renting a Knight K10 piano --- which has a beautifully bright and deep tone --- and practising daily.
Joined by Wilson Chen \zhname{WC}, Yun-Ju Chen \zhname{YJC}, Fu-Lien Hsieh \zhname{FLH}, Tao-Hsin Chang \zhname{THC}, and Duen-Wei Hsu \zhname{DWH}, we had three unforgettable evenings which started with dinner, followed by thirty minutes of my piano playing (mostly Chopin's works), and ending with a cosy chat going on late into the night.
Special thanks to Pan for keeping me company --- in particular, patiently responding to my endless (and sometimes pointless) Skype messages and travelling with me to London for concerts.

The concerts.
It's really a privilege to be able to live near London and regularly go to the superb concerts at the Southbank Centre, notably the performances of London Philharmonic Orchestra and Philharmonia Orchestra at the Royal Festival Hall.
The lively culture of Southbank Centre is pleasant and healing, and the amazing musicians showed me how natural it can be to strive for beauty and perfection.

All of these would probably not have been possible without the generous financial support from the University's Clarendon Fund, to which the College also contributes.
It is thanks to this support that I can concentrate on my study and even live a high-quality life (with so much music).
I was also supported by the UK Engineering and Physical Sciences Research Council project ``Reusability and Dependent Types'', so I could travel to conferences and workshops without having to worry about where to get funding.

Special thanks to Liang-Ting Chen \zhname{LTC} for numerous intellectual discussions, either on Facebook or in person, and hosting me several times in Birmingham.
I'd also like to give special thanks to Shin-Cheng Mu \zhname{SCM}\kern1pt\zhname{suffix}, who led me into programming languages research, and brought me to Oxford in 2008 to see what academia is like for the first time.
Thanks also go to Justin Chiu \zhname{JC}, Ting-Sung Hsieh \zhname{TSH}, Ching-Hao Wang \zhname{CHW}, and Wei-Jin Zheng \zhname{WJZ} for regularly chatting with me on Facebook and Skype, hosting me in the US, and crossing the Atlantic Ocean to visit me in Oxford.

My parents Jung-Feng Ko \zhname{JFK} and Hsiu-Ching Tsai \zhname{HCT} have always deeply cared for my well-being and provided unconditional support.
It is always heartwarming to see them on Skype every Sunday or when I'm back home in Taiwan, taking a break from the exhausting work.
I'm always filled with delicious food while I'm home, either by my mother's excellent cooking or going to various restaurants, many of which are carefully chosen by my older sister Lien-Fang Ko \zhname{LFK}.

On 18 March, when I was working on the last part of this dissertation, news about the Sunflower Movement in Taiwan swept through Facebook, causing quite some disruptions to my write-up plan.
I have my doubts about nationalism, but I'm happy with the idea when the name Taiwan is simply used to collectively refer to the people on the island, many of whom have helped me as I grow up and become who I am.
My experience in the UK has already shown how much advancement Taiwan still needs, and the Sunflower Movement revealed how urgently the advancement should happen.
Perhaps I will be able to contribute towards that advancement, as a big thanks to Taiwan.

\vspace*{8pt}
\hfill Josh Ko \zhname{JK}\\[-1pt]\null\hfill Oxford \& Changhua, 2014

\end{document}
