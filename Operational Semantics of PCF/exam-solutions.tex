\documentclass{article}
\usepackage{mathtools}
\usepackage{geometry}
\usepackage{amssymb}
\usepackage{amsthm}
\newcommand{\sem}[1]{[\![#1]\!]}
\title{Semantics of Functional Programming: Sample Solution}
\author{Liang-Ting Chen and Tyng-Ruey Chuang}
\date{11 July 2014}
\begin{document}
\maketitle

\begin{enumerate}
  \item (20\%) Which of the following posets are cpos? Why?
    \begin{enumerate}
      \item The set $(\mathbb{N}, \leq)$
        of natural numbers with the natural ordering. 
        \begin{proof}
          No. The set $\mathbb{N}$ of all natural numbers is directed, because
          for every two elements $x$ and $y$, either $x \leq y$ or $y \leq x$.
          However, there is no upper bound of $\mathbb{N}$ in $\mathbb{N}$.
        \end{proof}
      \item $(\mathbb{N} \cup \{\infty\}, \leq')$ with the natural ordering
        i.e.\ $n \leq' \infty$ for every $n \in \mathbb{N} \cup \{\infty\}$
        and $n \leq' m$ if $n \leq m$ for $m \in \mathbb{N}$. 
        \begin{proof}
          Yes.
          \begin{enumerate}
            \item $0$ is the element satisfying $0 \leq n$ with $n \in
              \mathbb{N}\cup\{\infty\}$, i.e.\ the bottom element. 
            \item We claim that every directed subset has the least upper
              bound. Let $D$ be a directed subset. If $D$ is finite, then by
              induction there is an upper bound $z$ of $D$ \emph{in} $D$ so that
              $z$ is the least upper bound. 
              If $D$ is infinite, then every element $x \in \mathbb{N}$ is not
              an upper bound. Otherwise $D$ is finite, since there are only
              finitely many elements below $x$. 
              By definition, $\infty$ is the upper bound of $D$,
              and any element $x$ with $\infty \leq x$ must be $\infty$. 
          \end{enumerate}

        \end{proof}
    \end{enumerate}
  \item (30\%) Evaluate the following terms using $\leadsto$ step by step
    \emph{without} formal derivations.
    \begin{enumerate}
      \item $(\lambda x.\, \lambda y.\, \lambda z.\, z) \;(\mathtt{Y} n.\,
        n)\;\underline{3}\;\underline{1}$
        \begin{proof}[Answer]
          $(\lambda x.\, \lambda y.\, \lambda z.\, z)
          \;(\mathtt{Y} n.\, n)\;\underline{3}\;\underline{1}
          \leadsto (\lambda y.\,\lambda z.\, z)\;\underline{3}\;\underline{1}
          \leadsto (\lambda z.\, z)\;\underline{1}
          \leadsto \underline{1}$
        \end{proof}
      \item $\left(\lambda z.\,\lambda n.\,\mathtt{ifz}(n; z; x.\,x)\right)\;
        \mathtt{ifz}(\underline{0}; \underline{1}; m.\, \mathtt{suc}\;m)$
        \begin{proof}[Answer]
          \begin{align*}
            & \left(\lambda z.\,\lambda n.\,\mathtt{ifz}(n; z; x.\,x)\right)\;
            \mathtt{ifz}(\underline{0}; \underline{1}; m.\, \mathtt{suc}\;m) \\
            \leadsto&(\lambda n.\, \mathtt{ifz}(n; \mathtt{ifz}(\underline{0};
            \underline{1}; m.\, \mathtt{suc}\;m);
            x. x)
          \end{align*}
        \end{proof}
    \end{enumerate}
  \item (30\%) Find the denotation of the following terms.
    \begin{enumerate}
      \item $z : \sigma \vdash \lambda x.\,\lambda y.\, x : \sigma_1 \to
        \sigma_2 \to \sigma_1$
        \begin{proof}[Answer]
          \begin{align*}
            &\sem{z : \sigma \vdash \lambda x.\,\lambda y.\, x : \sigma_1 \to
              \sigma_2 \to \sigma_1} \\
            ={} & \Lambda \sem{z : \sigma, x : \sigma_1
              \vdash \lambda y.\, x : \sigma_2\to\sigma_1} \\
            ={} & \Lambda (\Lambda
            \sem{z : \sigma, x : \sigma_1, y : \sigma_2 \vdash x : \sigma_1}) =
            \Lambda (\Lambda \pi_2)
        \end{align*}
        where $\pi_2 : 1 \times D_\sigma \times D_{\sigma_1} \times
        D_{\sigma_2}$ maps $(*, d, d_1, d_2) \in 1 \times D_\sigma \times
        D_{\sigma_1} \times D_{\sigma_2}$ to~$d_1$. 
      \end{proof}
      \item $x : \mathtt{nat}, y : \sigma, z : \sigma \vdash
        \mathtt{ifz}(x; y; m.\, z) : \sigma$
        \begin{proof}[Answer]
          For every $(*, d_1, d_2, d_3) \in 1 \times D_{\mathtt{nat}} \times D_\sigma
          \times D_\sigma$, we have
          \begin{align*}
            &\sem{x : \mathtt{nat}, y : \sigma, z : \sigma \vdash \mathtt{ifz}(x; y;
            m.\, z) : \sigma}\;(*, d_1, d_2, d_3) \\
          ={} & \mathit{ifz}(d_1, d_2, \mathit{const}_{d_3})
          \end{align*}
          or equivalently 
          \[
            \sem{x : \mathtt{nat}, y : \sigma, z : \sigma \vdash \mathtt{ifz}(x; y;
            m.\, z) : \sigma}\;(*, d_1, d_2, d_3) =
          \begin{cases}
            \bot & \text{if } d_1 = \bot \\
            d_2 & \text{if } d_1 = 0 \\
            d_3 & \text{if } d_1 > 0
          \end{cases}
          \]
        \end{proof}
    \end{enumerate}
  \item (10\%) Show that there is no term $\mathsf{V}$ satisfying
    $\mathtt{Y} x.\,x\Downarrow \mathsf{V}$.
    \textit{Hint}. Use the agreement of the one-step reduction and the big-step
    reduction, and the fact that $\leadsto$ is deterministic, i.e.\ 
    if $\mathsf{M} \leadsto \mathsf{M}_1$ and $\mathsf{M} \leadsto
    \mathsf{M}_2$ then $\mathsf{M}_1 = \mathsf{M}_2$. 
        \begin{proof}
          Suppose that there is a term $\mathsf{V}$ with $\mathtt{Y} x.\,
          x\Downarrow V$. By the agreement, $\mathtt{Y} x.\, x \leadsto^* V$
          and $V$ is a value.  By ($\leadsto$-fix),
          $\mathtt{Y} x.\,x\leadsto \mathtt{Y} x.\, x$
          and by determinacy every term $\mathsf{V}$ with $\mathtt{Y} x.\, x \leadsto V$ 
          must be $\mathtt{Y} x.\,x$ 
          By induction, every term $\mathsf{V}$ with $\mathtt{Y} x.\, x
          \leadsto^*\mathsf{V}$ must be $\mathtt{Y} x.\, x$. 

          However, $\mathtt{Y}x.\,x$ is not a value, so we reach a contradiction. 
        \end{proof}
        \begin{proof}
          Alternatively, we do induction on the derivation of $\mathtt{Y}x.\,x
          \Downarrow \mathsf{V}$, and the only case is ($\Downarrow$-fix), i.e. 
          to have $\mathtt{Y}x.\, x\Downarrow \mathsf{V}$ we must have
          ${x[\mathtt{Y}x.\, x/x] \Downarrow \mathsf{V}}$ as a premise
          satisfying the induction hypothesis. That is, there is no term
          $\mathsf{V}$ satisfying $\mathtt{Y} x.\,x \Downarrow \mathsf{V}$. 
        \end{proof}
  \item (10\%) Find a monotonic function
    $f : \mathbb{N} \cup \{\infty\}  \to \mathbb{N} \cup \{\infty\}$
    which is \emph{not} continuous. You have to give an example
    such that $f({\mbox{\Large $\sqcup$}} S) \neq
    {\mbox{\Large $\sqcup$}}_{x \in S} f(x)$.
        \begin{proof}[Answer]
          Let $f(n) = 0$ for $n < \infty$ and $f(\infty) = \infty$. 
          It is clear that $f$ is a monotonic function. 
          Since $\infty$ is the greatest element of $\mathbb{N}$,
          $\mbox{\Large $\sqcup$} \mathbb{N} = \infty$.
          Thus, $f(\mbox{\Large $\sqcup$} \mathbb{N}) = \infty
          \neq 0 = \mbox{\Large $\sqcup$}_{n \in \mathbb{N}} f(n)$.
        \end{proof}
\end{enumerate}
\end{document}
